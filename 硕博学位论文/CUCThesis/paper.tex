\documentclass{CUCThesis}

% 使用该宏包可以为参考文献、图表、公式添加超链接。但是可能会影响观感,可根据需要自行选择是否使用。
% 默认样式:使用方框标识超链接,很多期刊论文中有用
\usepackage[]{hyperref}
% 简化样式:不用方框,看起来会简洁一点,有各种颜色标识不同的类别
% \usepackage[colorlinks,linkcolor=red,anchorcolor=blue,citecolor=green]{hyperref}
% 保险样式:都是黑色的超链接,可以点击
% \usepackage[colorlinks,linkcolor=black,anchorcolor=black,citecolor=black,urlcolor=black]{hyperref}

% 指定宋体的格式为Windows下的宋体,需要配套的字体文件
\setCJKmainfont{SIMSUN.TTC}
\setCJKfamilyfont{heiti}{SimHei.ttf}
% 指定图片文件的默认目录
\graphicspath{{figures/}}

% 参考文献文件位置
\addbibresource{ref.bib}

% 论文参数设置
\setup{
  % 论文类别【博士学位论文/硕士学位论文/硕士专业学位论文/同等学力申请硕士学位论文】
  Type = 硕士学位论文,
  % 页眉 【中国传媒大学硕士学位论文/中国传媒大学博士学位论文】
  Header = 中国传媒大学硕士学位论文,
  % 分类号
  Classification = (按中国图书分类法),
  % 密级
  Security = (注明密级与保密期限),
  % 学号
  StudentID = 202120081000000,
  % 中文论文题目(封面)
  TitleinCover = \customtitle{基于 \LaTeX 的,硕博学位论文模板制作研究},
  % 英文论文题目(封面)
  TitleinCoverEN = \customtitle{Research on the Production of a LaTeX-based , {Master’s , and Doctoral Thesis Template}, CUC },
  % 中文论文题目(正文)
  TitleinBody = 基于 \LaTeX 的硕博学位论文模板制作研究,
  % 英文论文题目(正文)
  TitleinBodyEN = {Research on the Production of a LaTeX-based , {Master’s , and Doctoral Thesis Template}, CUC},
  % 申请人姓名
  Author = 王小帅,
  % 指导教师
  Supervisor = 蒋大帅,
  % 专业名称
  Major = \LaTeX 模板研发专业,
  % 研究方向
  Research = 学位论文研发,
  % 所在学院
  College = 信息与通信工程学院,
  % 提交日期
  Date = 2022年7月8日,
}

\setcounter{secnumdepth}{4}
\begin{document}

% ==========================================================
% ======================论文前置部分==========================
% ==========================================================
% 封面
\coverpage

% 独创性声明【这里可能要替换为扫描版PDF】
\copyrightpage

% 中英文摘要
% 中文摘要
\begin{cnabstract}[中国传媒大学;学位论文;\LaTeX 模板;研究生;博士生]
  摘要包括中文摘要和英文摘要两部分。摘要是论文内容的总结概括,应简要说明论文的研究目的、基本研究内容、研究方法、创新性成果及其理论与实际意义,突出论文的创新之处。不宜使用公式、图表,不标注引用文献。硕士论文摘要的字数一般为400-600 字,博士论文摘要的字数一般为800-1200字。英文摘要应与中文摘要内容相对应(见附件3:学位论文英文摘要版式)。摘要最后另起一行,列出3-5个关键词。关键词应体现论文特色,简要凝练并具有语义性,在论文中有明确的出处。
\end{cnabstract}

% 英文摘要
\begin{enabstract}[Communication University of China; Thesis; \LaTeX Template; Graduate; Doctor]
  The abstract consists of a Chinese abstract and an English abstract. It summarizes the main content of the thesis and should briefly state the research purpose, basic research content, research methods, innovative achievements, theoretical and practical significance of the thesis, highlighting its innovative aspects. The use of formulas and tables is not recommended, and references should not be cited. The word count for the abstract of a Master's thesis is generally 400-600 words, while that for a Ph.D. thesis is generally 800-1200 words. The English abstract should correspond to the content of the Chinese abstract (see Appendix 3: Format for the English Abstract of a Thesis). At the end of the abstract, 3-5 keywords should be listed on a separate line. The keywords should reflect the characteristics of the thesis, be concise and semantically meaningful, and have clear sources in the thesis.
\end{enabstract}

% 目录
\vspace*{-5em}
\tableofcontents
\thispagestyle{frontmatterstyle}
\addcontentsline{toc}{chapter}{目 \quad 录}
\clearpage

% 插图和附表清单
\listoffiguresandtables

% ==========================================================
% ======================论文主体部分==========================
% ==========================================================
% 从第一页开始 恢复阿拉伯数字页码
\pagenumbering{arabic}

% 正文写作,学位论文篇幅较长,建议分章节写作,每章节一个文件
\chapter{引言}
  引言内容 如:本文是为了研究……的问题,……,为此……,本文的结构安排如下:
  抱歉,我的建议没有帮助你解决问题。我再次检查了我的回答,发现我之前的建议是正确的。如果你在使用上面给出的最小工作示例时仍然遇到问题,那么可能是由于你的LaTeX编译器或系统环境配置不当导致的。
  \section{内容一}
  \begin{itemize}
    \item 第一章……
    \item 第二章……
    \item 第三章……
    \item 第四章……
  \end{itemize}

  \begin{figure}
    \centering
    \includegraphics[width=0.5\linewidth]{example-image-a.pdf}
    \caption{示例图片标题}
    \label{fig:example}
  \end{figure}


  \subsection{内容一的子内容一}
  This is a part of the paper.

    \begin{table}
    \centering
    \caption{三线表示例}
    \begin{tabular}{ll}
      \toprule
      文件名          & 描述                         \\
      \midrule
      thuthesis.dtx   & 模板的源文件,包括文档和注释 \\
      thuthesis.cls   & 模板文件                     \\
      thuthesis-*.bst & BibTeX 参考文献表样式文件    \\
      \bottomrule
    \end{tabular}
    \label{tab:three-line}
  \end{table}
  \clearpage
  \subsection{内容一的子内容二}
  This is a part of the paper.

  \begin{equation}
    E=mc^2
    \label{eq:example}
  \end{equation}

  \section{内容二}
  \subsection{内容二的子内容一}
  This is a part of the paper.
  \subsection{内容二的子内容二}
  适当引用文献\cite{rengongzhinengjianshi},适当引用文献\cite{zhongguozhexueshi},,适当引用文献\cite{jiqixuexi},适当引用文献\cite{vaswani_attention_2017},序号与本报告最后的参考文献一致。

\chapter{引用文献的标注}

  目前使用的后端为biber,编译方式为xelatex-biber-xelatex-xelatex
  
  
  \section{顺序编码制}
  
  在顺序编码制下,默认的命令同一样,序号置于方括号中,
  引文页码会放在括号外。
  统一处引用的连续序号会自动用短横线连接。
  
  \begin{tabular}{l@{\quad$\Rightarrow$\quad}l}
    \verb|\cite{zhangkun1994}|               & \cite{zhangkun1994}               \\
    \verb|\citet{zhangkun1994}|              & \citet{zhangkun1994}              \\
    \verb|\citep{zhangkun1994}|              & \citep{zhangkun1994}              \\
    \verb|\cite[42]{zhangkun1994}|           & \cite[42]{zhangkun1994}           \\
    \verb|\cite{zhangkun1994,zhukezhen1973}| & \cite{zhangkun1994,zhukezhen1973} \\
  \end{tabular}


  注意,引文参考文献的每条都要在正文中标注
  \cite{zhangkun1994,zhukezhen1973,dupont1974bone,zhengkaiqing1987,%
    jiangxizhou1980,jianduju1994,merkt1995rotational,mellinger1996laser,%
    bixon1996dynamics,mahui1995,carlson1981two,taylor1983scanning,%
    taylor1981study,shimizu1983laser,atkinson1982experimental,%
    kusch1975perturbations,guangxi1993,huosini1989guwu,wangfuzhi1865songlun,%
    zhaoyaodong1998xinshidai,biaozhunhua2002tushu,chubanzhuanye2004,%
    who1970factors,peebles2001probability,baishunong1998zhiwu,%
    weinstein1974pathogenic,hanjiren1985lun,dizhi1936dizhi,%
    tushuguan1957tushuguanxue,aaas1883science,fugang2000fengsha,%
    xiaoyu2001chubanye,oclc2000about,scitor2000project%
  }。

  \subsection{三级标题}

  \subsubsection{四级标题}

% ==========================================================
% ======================论文结尾部分==========================
% ==========================================================

% 参考文献,目前使用的后端为biber,编译方式为xelatex-biber-xelatex-xelatex
\begin{reference}
  \printbibliography[heading=none]
\end{reference}

% 附录

% 索引

% 作者简历及在学期间所取得的科研成果
\include{data/resume}

% 致谢
\begin{acknowledgements}
  首先,我想由衷地感谢我的导师[指导老师或主管的姓名],在整个研究过程中,他/她给予了我无私的指导和支持。导师的专业知识、鼓励和建设性的批评让我在研究中不断成长,让我的论文最终得以完成。

  此外,我还要感谢我的论文评审委员会成员[列出评审委员会成员的姓名],感谢他们在论文评审过程中给予的宝贵建议和意见,让我对自己的研究有了更深刻的认识和理解。
  
  我要感谢我的同事和朋友[列出同事和朋友的姓名],他们在我的研究中给予了我精神和智力上的支持。他们的理解、鼓励和激励一直以来都是我前进的动力和信心来源。
  
  感谢[列出资助机构或组织的名称]的大力支持,他们的经济援助让我得以专心致力于研究工作,让我的研究更加深入和完善。
  
  最后,我要感谢我的家人[列出家人的姓名],感谢他们一直以来的陪伴、支持和鼓励。他们的无私爱心和关心,让我在研究生活中充满了力量和动力。
  
  感谢你们所有人的帮助和支持,没有你们,我无法完成这个学术研究。再次感谢!
\end{acknowledgements}

\end{document}