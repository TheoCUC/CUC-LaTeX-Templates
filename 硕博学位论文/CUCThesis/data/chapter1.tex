\chapter{引言}
  引言内容 如:本文是为了研究……的问题,……,为此……,本文的结构安排如下:
  抱歉,我的建议没有帮助你解决问题。我再次检查了我的回答,发现我之前的建议是正确的。如果你在使用上面给出的最小工作示例时仍然遇到问题,那么可能是由于你的LaTeX编译器或系统环境配置不当导致的。
  \section{内容一}
  \begin{itemize}
    \item 第一章……
    \item 第二章……
    \item 第三章……
    \item 第四章……
  \end{itemize}

  \begin{figure}
    \centering
    \includegraphics[width=0.5\linewidth]{example-image-a.pdf}
    \caption{示例图片标题}
    \label{fig:example}
  \end{figure}


  \subsection{内容一的子内容一}
  This is a part of the paper.

    \begin{table}
    \centering
    \caption{三线表示例}
    \begin{tabular}{ll}
      \toprule
      文件名          & 描述                         \\
      \midrule
      thuthesis.dtx   & 模板的源文件,包括文档和注释 \\
      thuthesis.cls   & 模板文件                     \\
      thuthesis-*.bst & BibTeX 参考文献表样式文件    \\
      \bottomrule
    \end{tabular}
    \label{tab:three-line}
  \end{table}
  \clearpage
  \subsection{内容一的子内容二}
  This is a part of the paper.

  \begin{equation}
    E=mc^2
    \label{eq:example}
  \end{equation}

  \section{内容二}
  \subsection{内容二的子内容一}
  This is a part of the paper.
  \subsection{内容二的子内容二}
  适当引用文献\cite{rengongzhinengjianshi},适当引用文献\cite{zhongguozhexueshi},,适当引用文献\cite{jiqixuexi},适当引用文献\cite{vaswani_attention_2017},序号与本报告最后的参考文献一致。
